% Notation as used in Robotics, Vision and Control textbook
%
% Shorthand and abbreviations
%   \hom -> homogeneous transform
%   \Hom -> Homogeneous transform
%   \etal
%   \eq{e} -> (e) which is a reference to equation e, eg. \ref{e}
%   \fnumber -> f-number
%   \Mlab -> MATLAB
%   \comma      comma + a short space
%   \tsub{k}   display timestep subscript in angle brackets <k>
%   \cspace  configuration space, curly C
%
% Linear algebra notation
% =================
%
%  A pre-superscript is used to indicate the reference coordinate frame.  This can be created using the macro
%
%  \presup{A}
%
% Points
% --------
%
%  Points are displayed in ???
%
%   \point[f]{P}
%   \coord{x}{y}    (x, y)
%
% Abstract pose
% ==========
%
% \pose          xi
% \estpose     xi with a hat
% \posedot     greek letter nu for spatial velocity
%
% \pose[R]          xi with a pre-superscript
% \estpose[R]     xi with a hat and a pre-superscript
% \posedot[R]     greek letter nu for spatial velocity with a pre-superscript
%
% Vectors
% ----------
%
% Vectors are displayed in bold italic font
%
%   \vec{x}     a vector
%   \vec[f]{x}  a vector with a coordinate frame
%
%   \dvec[f]{x}  a dotted vector 
%   \ddvec[f]{x}  a double dotted vector 
%   \vech[f]{x} a homogeneous vector (tilde above)
%   \vecb[f]{x} a vector with bar 
%   \tvec[f]{x} a homogeneous vector with a tilde
%   \hvec[f]{x} an estimated vector with a hat
%
%  if [f] not provided no presup is displayed
%
%
% Matrices
% -----------
%
%  Matrices are displayed in bold font
%
%   \mat{x}     a matrix
%   \mat[f]{x}  a matrix with a coordinate frame
%   \dmat[f]{x}  a matrix derivative
%   \sk{x}  -> S(x) skew symmetric matrix

% Quaternion
%
%   \q            a quaternion, q with a bubble on top
%
% Poses
%
%   \pose       pose
%   \estpose       estimated pose, with hat
%
%   \pose[f]    pose with respect to a frame
%   \pose        abstract pose
%   \posedot        derivative of abstract pose
%
% Lie Groups
% ========
%
% displayed in Roman font
%
%   \SO{n}   SO(n)
%   \SE{n}   SE(n)
%   \so{n}     so(n)
%   \se{n}     se(n)
%
% MATLAB code
% ===========
%
% Display a block of code in blue fixed-width font
%
%   \begin{Code}
%   \end{Code}
%
% other environments include
%   CodeSmall   uses a small font, useful for very wide lines
%   CodeNum    numbers the lines of code
%
% \var{} displays a variable in blue fixed-width font 
% \Mlab  displays MATLAB with the all letters bar the first in small caps font
%
% Units
% ====
% 
% Always set in Roman font with preceding thin space, can be used in math or LR mode.
%
%   \unit{}  eg. 20\unit{lx/m^2}.  
%   \Hz
%   \ms
%   \us   microseconds using mu
%   \nm
%   \deg  degrees shown with a bubble
%
% General shorthand notation
% =====================
%
%   \sci{5}{-2} format in scientific notation, 5 x 10^{-2}
%
%  \vector{x}{y}{z}    (x, y, z)
%  \comma  a comma followed by space
%  \cframe{a}  Coordinate frame {a}
%
%   \eq{ref}    reference an equation, in parentheses
%   \etal       et al.
%
%   \ba    begin eqnarray
%   \ea    end eqnarray
%
%   \hom    homogeneous transform
%   \Hom    Homogeneous transform
%
% Some common log-like functions
%
%   \floor
%   \ceil
%   \ord
%   \det
%   \trace  -> tr
%
%

\usepackage{amsmath, amssymb}

\usepackage{xifthen}
\usepackage{color}

\typeout{------ start of RVC notation ------}
\newcommand{\f}{f}

%

%
\renewcommand{\hom}{homogeneous transform}
\newcommand{\Hom}{Homogeneous transform}

\newcommand{\homeq}{\simeq}

\newcommand{\xstrut}[1]{\rule{0ex}{#1ex}}

\newcommand{\tsub}[1]{{\scriptstyle{\langle #1 \rangle}}}

\newcommand{\hstrut}{\rule{0mm}{7mm}}
\newcommand{\ba}{\begin{eqnarray}}
\newcommand{\ea}{\end{eqnarray}}

\newcommand{\eq}[1]{(\ref{#1})}
\newcommand{\etal}{{et al.}}
\newcommand{\fnumber}{\textit{f}-number}

%
% some common functions
%
%   \floor
%   \ceil
%   \ord
%   \det
%   \sk{}   skew symmetric matrix
\newcommand{\floor}{\mbox{floor}}
\newcommand{\ceil}{\mbox{ceil}}
\newcommand{\ord}{\mbox{Ord}}
\newcommand{\vex}{\mbox{vex}}

\renewcommand{\det}[1]{{\mbox{det}(\mat{#1})}}
\newcommand{\trace}[1]{{\mbox{tr}(\mat{#1})}}


%
% use as \sci{5}{-2} for formatting 5e-2
%
\newcommand{\sci}[2]{\mbox{$#1\!\cdot\!10^{#2}$}}


%
% macros to display units in math or LR mode, 20\unit{lx/m^2}
%
%   \unit{}
%   \Hz
%   \ms
%   \us
%   \deg
%
\newcommand{\unit}[1]{\mbox{$\mathrm{\,#1}$}}
\newcommand{\Hz}{\unit{Hz}}
\newcommand{\kHz}{\unit{Hz}}
\newcommand{\MHz}{\unit{Hz}}
\newcommand{\ms}{\unit{ms}}
\newcommand{\us}{\unit{\mu s}}
\renewcommand{\deg}{\mbox{${}^\circ$}}


\newcommand{\nm}{\unit{nm}}
\newcommand{\Dsixfive}{$\rm D_{65}$}

\newcommand{\R}{{\bf R}}
\newcommand{\zero}{{\bf 0}}

%
% notational shorthand (for consistency)
%   \q            a quaternion
%
%   \pose       pose
%   \pose[f]    pose with respect to a frame
%   \vec{x}     a vector
%   \vec[f]{x}  a vector with a coordinate frame
%   \dvec[f]{x}  a dotted vector with a coordinate frame
%   \ddvec[f]{x}  a double dotted vector with a coordinate frame
%   \vech{x}    a homogeneous vector
%   \vech[f]{x} a homogeneous vector with a coordinate frame
%   \vecb{x}    a vector with bar
%   \vecb[f]{x} a vector with bar and a coordinate frame
%   \hvec{x}    a vector with a hat
%
%   \mat{x}     a matrix
%   \mat[f]{x}  a matrix with a coordinate frame
%
%   \point[f]{P}
%   \frame[f]{P}{s}
%   \frameh[f]{P}{s}   \hat{P}
%   \frameb[f]{P}{s}   \bar{P}
%
%   \comma      comma + a short space
%   \coord{x}{y}    (x, y)
%   \vector{x}{y}{z}    (x, y, z)
%  \pose        abstract pose
%  \posedot        derivative of abstract pose
\newcommand{\im}[1]{\mathbf{#1}}
\newcommand{\presup}[1]{\,{}^{\scriptscriptstyle #1}\!}
\newcommand{\relval}[2]{\presup{#1}{#2}}
\newcommand{\pose}[1][ZZZZ]{\ifthenelse{\equal{#1}{ZZZZ}}{}{\presup{#1}}{\mathbf{\xi}}}
\newcommand{\estpose}[1][ZZZZ]{\ifthenelse{\equal{#1}{ZZZZ}}{}{\presup{#1}}{\mathbf{\hat{\xi}}}}
\newcommand{\hpose}[1][ZZZZ]{\ifthenelse{\equal{#1}{ZZZZ}}{}{\presup{#1}}{\hat{\mathbf{\xi}}}}
\newcommand{\posedot}[1][ZZZZ]{\ifthenelse{\equal{#1}{ZZZZ}}{}{\presup{#1}}{\mathbf{\nu}}}


\newcommand{\q}{\mathring{q}}

\DeclareMathAlphabet{\mathitbf}{OML}{cmm}{b}{it}
\newcommand{\twist}[2][ZZZZ]{\ifthenelse{\equal{#1}{ZZZZ}}{}{\presup{#1}}{\mathcal{S}}}
\renewcommand{\vec}[2][ZZZZ]{\ifthenelse{\equal{#1}{ZZZZ}}{}{\presup{#1}}{\mathitbf{#2}}}

\newcommand{\hvec}[2][ZZZZ]{\ifthenelse{\equal{#1}{ZZZZ}}{}{\presup{#1}}{\hat{\vec{#2}}}}
\newcommand{\tvec}[2][ZZZZ]{\ifthenelse{\equal{#1}{ZZZZ}}{}{\presup{#1}}{\tilde{\vec{#2}}}}
\newcommand{\evec}[2][ZZZZ]{\ifthenelse{\equal{#1}{ZZZZ}}{}{\presup{#1}}{\hat{\vec{#2}}}}
%
\newcommand{\dvec}[2][ZZZZ]{\ifthenelse{\equal{#1}{ZZZZ}}{}{\presup{#1}}{\dot{\vec{#2}}}}
\newcommand{\ddvec}[2][ZZZZ]{\ifthenelse{\equal{#1}{ZZZZ}}{}{\presup{#1}}{\ddot{\vec{#2}}}}

\newcommand{\hv}[1]{\ensuremath{\stackrel{\textstyle{#1}}{\textstyle{\sim}}}}
\newcommand{\vech}[2][ZZZZ]{\ifthenelse{\equal{#1}{ZZZZ}}{}{\presup{#1}}{\mathitbf{\tilde{#2}}}}
\newcommand{\vecb}[2][ZZZZ]{\ifthenelse{\equal{#1}{ZZZZ}}{}{\presup{#1}}{\bar{\underline #2}}}
\newcommand{\mat}[2][ZZZZ]{\ifthenelse{\equal{#1}{ZZZZ}}{}{\presup{#1}\,}{{\boldsymbol #2}}}
\newcommand{\hmat}[2][ZZZZ]{\ifthenelse{\equal{#1}{ZZZZ}}{}{\presup{#1}\,}{{\hat{\boldsymbol #2}}}}
\newcommand{\dmat}[2][ZZZZ]{\ifthenelse{\equal{#1}{ZZZZ}}{}{\presup{#1}\,}{{\dot{\boldsymbol #2}}}}
\newcommand{\emat}[2][ZZZZ]{\ifthenelse{\equal{#1}{ZZZZ}}{}{\presup{#1}\,}{\hat{\boldsymbol #2}}}
\newcommand{\matfn}[3][ZZZZ]{\ifthenelse{\equal{#1}{ZZZZ}}{}{\presup{#1}}{{\mat{#2}}\left(#3\right)}}
\newcommand{\Rt}[2][ZZZZ]{\ifthenelse{\equal{#1}{ZZZZ}}{}{\presup{#1}}{{\bf R}\left(#2\right)}}


%\newcommand{\homt}[3]{{}^{#1}{\bf #3}_{#2}}
%\renewcommand{\frame}[2]{{}^{#1}{#2}}

\newcommand{\cframe}[1]{\{#1\}}
\newcommand{\cspace}{{\cal C}}
\newcommand{\point}[2][ZZZZ]{\ifthenelse{\equal{#1}{ZZZZ}}{}{\presup{#1}}{\mathbf{\mathrm{#2}}}}

\newcommand{\comma}{,\;}
\newcommand{\coord}[2]{\left( #1 \comma #2 \right)}
\renewcommand{\vector}[3]{\left( #1 \comma #2 \comma #3 \right)}



% Matlab documentation macros
\newfont{\School}{pncr}
\newfont{\eightTR}{pncr at 8pt}
%\newfont{\vtt}{pcrr}

\newcommand{\fullstop}{\,\,\,.}
\newcommand{\termsep}{\,\,\,,}
\typeout{------ end of notation ------}

\newcommand{\del}[1]{\delta_{#1}}
\newcommand{\vdel}[1]{\vec{\delta}_{#1}}

%=====================================================================
%  MATLAB code support
%=====================================================================

\usepackage{color}
% Matlab code example environment
%   for code examples.
%   each executable line must begin with >>
%   comments start with !
%   lines starting !>> are executed during parsing, but dont print
%   
%   \begin{Code}
%   \end{Code}
\usepackage{fancyvrb}
\fvset{formatcom=\color{blue},fontseries=c,fontfamily=courier,xleftmargin=4mm,commentchar=!}

\DefineVerbatimEnvironment{Code}{Verbatim}{formatcom=\color{blue},fontseries=c,fontfamily=courier,fontsize=\footnotesize,xleftmargin=4mm,commentchar=!}

\DefineVerbatimEnvironment{CodeSmall}{Verbatim}{formatcom=\color{blue},fontseries=c,fontfamily=courier,fontsize=\scriptsize,xleftmargin=1mm,commentchar=!}

\DefineVerbatimEnvironment{CodeNum}{Verbatim}{numbers=left,numbersep=4pt,formatcom=\color{blue},fontseries=c,fontfamily=courier,fontsize=\footnotesize,xleftmargin=4mm}

%\var{}
%\Mlab

\newcommand{\var}[1]{{\color{blue}\Verb+#1+}}
\newcommand{\model}[1]{\index{code}{#1@\textit{#1}}\ifthenelse{\boolean{draft}}{{\color{green}\Verb+#1+}}{\Verb+#1+}}
\newcommand{\block}[1]{\ifthenelse{\boolean{draft}}{{\color{green}\Verb+#1+}}{\textsf{#1}}}

%\newcommand{\func}[1]{{\color{blue}\Verb+#1+}}
\newcommand{\func}[2][ZZZZ]{\ifthenelse{\equal{#1}{ZZZZ}}{\index{code}{#2}}{\index{code}{#1}}\ifthenelse{\boolean{draft}}{{\color{green}\Verb+#2+}}{\Verb+#2+}}

\newcommand{\methodb}[2]{\index{code}{#1@\textbf{#1}!.#2}\ifthenelse{\boolean{draft}}{{\color{magenta}\Verb+#1.#2+}}{\Verb+#1.#2+}}
\newcommand{\method}[2]{\index{code}{#1@\textbf{#1}!.#2}\ifthenelse{\boolean{draft}}{{\color{magenta}\Verb+#2+}}{\Verb+#2+}}
\newcommand{\class}[1]{\index{code}{#1@\textbf{#1}}\ifthenelse{\boolean{draft}}{{\color{cyan}\Verb+#1+}}{\Verb+#1+}}
\newcommand{\property}[1]{\index{property}{#1}\ifthenelse{\boolean{draft}}{{\color{cyan}\Verb+#1+}}{\Verb+#1+}}

%\newcommand{\Mlab}{{M\textsc{ATLAB}}}
\newcommand{\MATLAB}{MATLAB\textsuperscript{\textregistered}}

\newcommand{\SE}[1]{\ensuremath{\mathrm{{\bf SE}(#1)}}}
\newcommand{\SO}[1]{\ensuremath{\mathrm{{\bf SO}(#1)}}}
\newcommand{\se}[1]{\ensuremath{\mathrm{{\bf se}(#1)}}}
\newcommand{\so}[1]{\ensuremath{\mathrm{{\bf so}(#1)}}}
\newcommand{\isk}[1]{\vee\left( #1\right)}
\newcommand{\iskx}[1]{\vee_{\times}\left( #1\right)}
\newcommand{\skx}[1]{\left[#1\right]_{\times}}
\newcommand{\sk}[1]{\left[#1\right]}
%\newcommand{\norm}[1]{\Vert #1 \Vert}
\newcommand{\Ad}{\mbox{Ad}}
\newcommand{\ad}{\mbox{ad}}
